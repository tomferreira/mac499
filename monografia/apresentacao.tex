\documentclass{beamer}
\usepackage[brazilian]{babel}
\usepackage[utf8]{inputenc}
\usepackage[T1]{fontenc}
\usepackage{amsmath}
\usepackage{graphicx}
\usepackage{subfigure}
\usepackage{multirow}
\usepackage{verbatim}
\mode<presentation>{\usetheme[darktitle,framenumber,compress]{UniversiteitAntwerpen}} 
\title{Análise e conversão de tabloides de promoções}
\author{Igor dos Santos Montagner \\ Orientador: Prof. Dr. Roberto Marcondes Cesar Junior }
\graphicspath{{imgs/}}

\AtBeginSection[]
{
   \begin{frame}
       \frametitle{Agenda}
       \tableofcontents[currentsection]
   \end{frame}
}

\begin{document}

\begin{frame}
\titlepage
\end{frame}

\section{Problema}

\begin{frame}
\frametitle{Problema}
Apesar do desenvolvimento dos meios eletrônicos a quantidade de papel usada ainda é grande; \pause
\begin{figure}[!ht]
    \includegraphics[scale=0.5]{papel1.jpg}
\end{figure}   
\end{frame}

\begin{frame}
\frametitle{Problema}
\begin{center}muito GRANDE; \end{center}
\begin{figure}[!ht]
    \includegraphics[scale=0.25]{papel2.jpg}
\end{figure}   
\end{frame}

\begin{frame}
\frametitle{Problema}

\begin{itemize}

\item Existe uma necessidade de buscar, indexar e armazenar documentos e informações de forma eficiente; \pause

\item Em um mercado concorrido e em busca de rapidez em suas operações, a conversão manual das informações contidas em papel está longe de ser ideal;

\end{itemize}

\end{frame}

\begin{frame}
\frametitle{Problema}

\begin{itemize}

\item No mercado de consumo de varejo, a rapidez na análise da concorrência pode ser um diferencial; \pause

\item Isto pode ser feito usando as informações presentes nos tabloides promocionais distribuídos nos estabelecimentos comerciais.

\end{itemize}

\end{frame}


\begin{frame}
    \frametitle{Problema}
    Porém existem muitos formatos de tabloides.
    \begin{figure}[!ht]
    \begin{center}
        \includegraphics[scale=0.4]{tabloides.png}
    \end{center}
    \end{figure}
\end{frame}

\subsection{Objetivo final}

\begin{frame}
    \frametitle{Objetivos}
    Os objetivos do trabalho são construir um sistema que:
    \pause
    \begin{itemize}
        \item produza uma listagem com os preços de cada produto encontrado nas páginas de um tabloide;
        \pause
        \item torne viável a análise de tabloides de diferentes estabelecimentos comerciais.
    \end{itemize}
\end{frame}

\begin{frame}
    \frametitle{Objetivos}
    
    \begin{columns}[c]
        \column{100pt} {
            \includegraphics[scale=0.18]{seg-final-o.png}
        }
        \column{3pt} {
            $\rightarrow$
        }
        \column{200pt} {
            Coca Cola normal ou PET 600ml ... 1,79
            
            Sorvete Nestlé vários sabores - 2L ... 11, 80
            
            $\dots$
            
            $\dots$
        }
    \end{columns} 

\end{frame}

\section{Solução proposta}

\begin{frame}
    \frametitle{Solução proposta}
    Foi fixado somente um formato de tabloide:
    \begin{figure}[!ht]
    \begin{center}
        \includegraphics[scale=0.18]{seg-final-o.png}
    \end{center}
    \end{figure}
\end{frame}

\begin{frame}
    \frametitle{Solução proposta}
    Para poder lidar com a constante modificação de \emph{layout} que ocorre nos tabloides, o trabalho de análise foi dividido nas seguintes etapas:
    \pause
    \begin{figure}[!ht]
    \begin{center}
        \includegraphics[scale=0.4]{flow-H.png}
    \end{center}
    \end{figure}
    \pause
    Esta divisão permite adicionar suporte a um novo tipo de tabloide sobrescrevendo as etapas que falham se usado um analisador para outro tipo.
\end{frame}



\subsection{Atenção visual e mapa de saliências}

\begin{frame}
\frametitle{Atenção visual e mapa de saliências}
	\begin{itemize}
	    \item É natural que um leitor, ao observar uma imagem, examine algumas partes primeiramente e com maior atenção que outras. \pause 
	    \item Em tabloides promocionais este fato é mais relevante, já que não existe uma ordem natural de leitura. \pause
	    \item Este fenômeno, chamado de \emph{atenção seletiva}, é descrito em \cite{survey10}, juntamente com sistemas de computação que o utilizam para analisar imagens.
    \end{itemize}
\end{frame}

\begin{frame}
\frametitle{Atenção visual e mapa de saliências}
No trabalho \cite{Ma e Zhang03}, o contraste é usado como medida principal de atenção.

\begin{figure}[!ht]
    \begin{center}
        \includegraphics[scale=.5]{fig1.png}
    \end{center}
\end{figure}
\end{frame}

\begin{frame}
\frametitle{Atenção visual e mapa de saliências}
Um \emph{mapa de saliências} é usado para medir a atenção visual em uma imagem. O contraste entre um pixel e sua vizinhança foi usado para a criação do mapa.


\begin{figure}[!ht]
    \begin{center}
        \includegraphics[scale=.15]{msf.png}
    \end{center}
\end{figure}
\end{frame}


\subsection{Segmentação do background}

\begin{frame}
\frametitle{Segmentação do \emph{background}}
O mapa de saliências da imagem é calculado e tudo o que não for foco de atenção na imagem será considerado como \emph{background}. 
    
\begin{figure}[!ht]
    \subfigure{\includegraphics[scale=.15]{ms-orig.png} } \pause
    \subfigure{\includegraphics[scale=.15]{msf.png}} \pause
    \subfigure{\includegraphics[scale=.15]{ms-fundo.png}}
\end{figure}

\end{frame}

\subsection{Detecção e recorte de produtos}

\begin{frame}
\frametitle{Detecção e recorte de produtos}

O tamanho de cada região não pertencente ao \emph{background} é usado para identificar produtos. \pause O recorte é feito a partir da detecção das linhas divisórias de produtos. \pause 
            
\begin{figure}[!ht]
    \subfigure{\includegraphics[scale=.15]{linhas-o1.png} } \pause
    \subfigure{\includegraphics[scale=.15]{linhas-l1.png} } \pause
    \subfigure{\includegraphics[scale=.15]{recorte-seg.png} }
\end{figure}


\end{frame}

\subsection{Detecção de textos e OCR }
\begin{frame}
\frametitle{Detecção de textos e OCR}
    \begin{itemize}
    \item A detecção de textos dentro de um recorte foi feita usando \emph{classificação supervisionada} com o classificador \emph{k-Vizinhos mais próximos}. \pause
    
    
    \item O conjunto de treinamento consistiu de 192 exemplos gerados a partir de 10 recortes de imagens. 
    \end{itemize}
\end{frame}

\begin{frame}
\frametitle{Detecção de textos e OCR}
        
    Foram extraídas 5 características:\pause
    \begin{itemize}
        \item largura\pause
        \item altura\pause
        \item aspecto\pause
        \item média da região no mapa de saliências\pause
        \item desvio padrão da região no mapa de saliências
    \end{itemize}\pause
     e usadas 3 classes: \emph{nome}, \emph{preço} e \emph{``qualquer outra coisa''}.
\end{frame}

\begin{frame}
\frametitle{Detecção de textos e OCR}
\begin{figure}[h!]
    \begin{center}
        \includegraphics[scale=0.5]{knn.png}
        \caption{Ilustração do processo de classificação com $k=5$. A classificação de {\bf x} como preto é feita segundo a classe mais comum entre os 5 vizinhos mais próximos \label{fig:knn}}
    \end{center}
\end{figure}
\end{frame}

\begin{frame}
\frametitle{Detecção de textos e OCR}
    \begin{figure}[!h]
            \subfigure{\includegraphics[scale=0.15]{class-rec.png}}
            \subfigure{\includegraphics[scale=0.15]{class-regioes.png}}
            \subfigure{\includegraphics[scale=0.15]{class-final.png}}
            \caption{Recorte de um produto, as regiões a serem classificadas e o resultado final da detecção de nomes de produtos (em vermelho) e preços (em azul)}
    \end{figure}
\end{frame}

\begin{frame}
\frametitle{Detecção de textos e OCR}
    A leitura dos caracteres foi feita usando o \emph{Tesseract OCR}.  \pause
    \begin{itemize} \item Cada região de texto é pré-processada antes da leitura dos caracteres para melhorar o reconhecimento. \end{itemize}

\end{frame}

\begin{frame}
\frametitle{Detecção de textos e OCR}
    \begin{table}[!h]
        \begin{center}
            \begin{tabular} {r c l}	
                \subfigure{\includegraphics[scale=.2]{textos-1.png}} &
                \subfigure{\includegraphics[scale=.2]{textos-2.png}} &
                \subfigure{\includegraphics[scale=.2]{textos-2-r.png}} \\
                \subfigure{\includegraphics[scale=.2]{textos-3.png}} &
                \subfigure{\includegraphics[scale=.2]{textos-4.png}} &
                \subfigure{\includegraphics[scale=.2]{textos-4-r.png}} \\
                \subfigure{\includegraphics[scale=.2]{textos-5.png}}&
                \subfigure{\includegraphics[scale=.2]{textos-6.png}}&
                \subfigure{\includegraphics[scale=.2]{textos-6-r.png}}
                
            \end{tabular}
        \end{center}
    \end{table}
    
\end{frame}

\section{Conclusões}
\begin{frame}
\frametitle{Conclusões}
    O trabalho mostrou a viabilidade da construção de um sistema analisador de tabloides promocionais e, com as etapas definidas é possível expandir o programa para outros \emph{layouts}. A teoria de \emph{atenção visual} foi essencial para a realização do trabalho e o mapa de saliências baseado em contrastes foi usado em todas as etapas do trabalho.
\end{frame}

\section{Agradecimentos}

\begin{frame}
\frametitle{Agradecimentos}
    Agradeço à Diretoria de Inovação do Ibope Media pelo apoio ao trabalho e a Celina Takemura pela ajuda nas soluções do trabalho e na elaboração do texto.
\end{frame}

\section{Referências}
\begin{frame}
    \frametitle{Referências}
    \begin{thebibliography}{9}
        \bibitem{survey10}
            S. Frintrop, E. Rome, H. I. Christensen. 
            \emph{Computational visual attention systems and their cognitive foundations: A survey}. 
            ACM Trans. Appl. Percept. 7, 1 (Jan. 2010), 1-39.
            2010.
	\bibitem{Ma e Zhang03}  
            Y. Ma, H. Zhang.
            \emph{Contrast-based image attention analysis by using fuzzy growing}.
            Proceedings of the Eleventh ACM international Conference on Multimedia
            Berkeley, CA, USA.
            2003.
        \bibitem{Liu.et.al07}
            H. Liu, S. Jiang, Q. Huang, C. Xu, W. Gao
            \emph{Region-based visual attention analysis with its application in image browsing on small displays}. 
            In Proceedings of the 15th international Conference on Multimedia  (Augsburg, Germany, September 25 - 29, 2007).
            2007.
            
    \end{thebibliography}
\end{frame}

\end{document}

