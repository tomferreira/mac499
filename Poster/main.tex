%%%%%%%%%%%%%%%%%%%%%%%%%%%%%%%%%%%%%%%%%
% Jacobs Landscape Poster
% LaTeX Template
% Version 1.1 (14/06/14)
%
% Created by:
% Computational Physics and Biophysics Group, Jacobs University
% https://teamwork.jacobs-university.de:8443/confluence/display/CoPandBiG/LaTeX+Poster
% 
% Further modified by:
% Nathaniel Johnston (nathaniel@njohnston.ca)
%
% This template has been downloaded from:
% http://www.LaTeXTemplates.com
%
% License:
% CC BY-NC-SA 3.0 (http://creativecommons.org/licenses/by-nc-sa/3.0/)
%
%%%%%%%%%%%%%%%%%%%%%%%%%%%%%%%%%%%%%%%%%

%----------------------------------------------------------------------------------------
%	PACKAGES AND OTHER DOCUMENT CONFIGURATIONS
%----------------------------------------------------------------------------------------

\documentclass[final]{beamer}

\usepackage[orientation=landscape,size=a2,scale=1.4]{beamerposter}
%\usepackage[scale=1.24]{beamerposter} % Use the beamerposter package for laying out the poster

\usetheme{confposter} % Use the confposter theme supplied with this template

\setbeamercolor{block title}{fg=ngreen,bg=white} % Colors of the block titles
\setbeamercolor{block body}{fg=black,bg=white} % Colors of the body of blocks
\setbeamercolor{block alerted title}{fg=white,bg=dblue!70} % Colors of the highlighted block titles
\setbeamercolor{block alerted body}{fg=black,bg=dblue!10} % Colors of the body of highlighted blocks
% Many more colors are available for use in beamerthemeconfposter.sty

%-----------------------------------------------------------
% Define the column widths and overall poster size
% To set effective sepwid, onecolwid and twocolwid values, first choose how many columns you want and how much separation you want between columns
% In this template, the separation width chosen is 0.024 of the paper width and a 4-column layout
% onecolwid should therefore be (1-(# of columns+1)*sepwid)/# of columns e.g. (1-(4+1)*0.024)/4 = 0.22
% Set twocolwid to be (2*onecolwid)+sepwid = 0.464
% Set threecolwid to be (3*onecolwid)+2*sepwid = 0.708

\newlength{\sepwid}
\newlength{\onecolwid}
\newlength{\twocolwid}
\newlength{\threecolwid}
\setlength{\sepwid}{0.024\paperwidth} % Separation width (white space) between columns
\setlength{\onecolwid}{0.22\paperwidth} % Width of one column
\setlength{\twocolwid}{0.464\paperwidth} % Width of two columns
\setlength{\threecolwid}{0.708\paperwidth} % Width of three columns
%-----------------------------------------------------------

\usepackage[brazil]{babel}
\usepackage[utf8]{inputenc}
\usepackage{graphicx}  % Required for including images
\graphicspath{{images/}}

\usepackage{booktabs} % Top and bottom rules for tables

%----------------------------------------------------------------------------------------
%	TITLE SECTION 
%----------------------------------------------------------------------------------------

\title{Classificação não-supervisionada hierárquica de artigos jornalísticos} % Poster title

\author{Cirillo Ribeiro Ferreira (cirillo.ferreira@usp.br) Orientador: Prof. Dr. Alair Pereira do Lago} % Author(s)

\institute{Instituto de Matemática e Estatística, Universidade de São Paulo - Trabalho de Formatura Supervisionado} % Institution(s)

%----------------------------------------------------------------------------------------

\begin{document}

\addtobeamertemplate{block end}{}{\vspace*{1ex}} % White space under blocks
\addtobeamertemplate{block alerted end}{}{\vspace*{1ex}} % White space under highlighted (alert) blocks

\setlength{\belowcaptionskip}{1ex} % White space under figures
\setlength\belowdisplayshortskip{1ex} % White space under equations

\begin{frame}[t] % The whole poster is enclosed in one beamer frame

\begin{columns}[t] % The whole poster consists of three major columns, the second of which is split into two columns twice - the [t] option aligns each column's content to the top

\begin{column}{\sepwid}\end{column} % Empty spacer column

\begin{column}{\onecolwid} % The first column

%----------------------------------------------------------------------------------------
%	OBJECTIVES
%----------------------------------------------------------------------------------------

\begin{alertblock}{Objetivos}

Este trabalho tem como objetivos:
\begin{itemize}
\item Estudos das principais classes de algoritmos para agrupamento de documentos textuais.
\item Criação de uma biblioteca para agrupamento de artigos jornalísticos disponíveis nos meios digitais.
\item Proposta e implementação do sistema hVINA (\textit{Hierarchical Viewer of News Articles}) para simplificar a interação entre o usuário e a biblioteca.
\end{itemize}

\end{alertblock}

%----------------------------------------------------------------------------------------
%	INTRODUCTION
%----------------------------------------------------------------------------------------

\begin{block}{Introdução}

Com a criação da internet e a popularização de seu uso como ferramenta de comunicação, houve uma explosão de informação que tornou muito difícil a classificação dos documentos produzidos e publicados nela de maneira manual. A área de classificação de documentos é de grande interesse e possui diversas aplicações práticas como classificação de \textit{spam}, identificação de idioma e análise de sentimento. Em especial, a classificação de artigos jornalísticos tem um enorme desafio devido à grande quantidade de novos documentos criados diariamente e a diversidade de temas abordados. %, especialmente em blogs, mas que carecem de melhor organização.

Para tal propósito serão utilizados algoritmos de aprendizagem não-supervisionadas, pois não necessitam de um conjunto de treinamento como entrada, permitindo o seu uso em conjuntos de dados bem variados, algo que é bem comum em artigos jornalísticos.

\end{block}

%------------------------------------------------

\begin{figure}
\includegraphics[width=0.5\linewidth]{fihc.png}
\caption{Exemplo de agrupamento feito pelo FIHC}
\end{figure}

%----------------------------------------------------------------------------------------

\end{column} % End of the first column

\begin{column}{\sepwid}\end{column} % Empty spacer column

\begin{column}{\twocolwid} % Begin a column which is two columns wide (column 2)

\begin{columns}[t,totalwidth=\twocolwid] % Split up the two columns wide column

\begin{column}{\onecolwid}\vspace{-.6in} % The first column within column 2 (column 2.1)

%----------------------------------------------------------------------------------------
%	MATERIALS
%----------------------------------------------------------------------------------------

\begin{block}{Análise de agrupamento}

Análise de agrupamento é uma classificação de padrão que emprega o processo de aprendizagem não-supervisionada e tem como objetivo o particionamento de objetos em grupos cujo membros sejam similares entre si e diferentes dos membros de outros grupos \cite{Jain:1999}.

%\begin{enumerate}
%\item Curabitur pellentesque dignissim
%\item Eu facilisis est tempus quis
%\end{enumerate}

\end{block}

\begin{block}{Algoritmos hierárquicos}

Os algoritmos da área de agrupamento são divididos geralmente em duas classes: Algoritmos planos e hierárquicos.
Os algoritmos hierárquicos são aqueles que geram uma árvore de grupos (\textit{cluster}). Uma estrutura que fornece mais informação, uma vez que as relações implícitas entre os grupos ficam mais evidentes \cite{Manning:2009}.
Há duas abordagens na criação da árvore, a primeira chamada de divisiva ou \textit{top-down} inicia a construção a partir da raiz até as folhas, a segunda chamada de aglomerativa ou \textit{bottom-up} inicia a construção das folhas à raiz.

\end{block}

%----------------------------------------------------------------------------------------
%	MATHEMATICAL SECTION
%----------------------------------------------------------------------------------------

\begin{block}{FIHC}

Foi implementado inicialmente na biblioteca o \textit{Frequent Itemset-based Hierarchical Clustering} (FIHC), que é um algoritmo para agrupamento hierárquico de documentos textuais \cite{Martin:2004} que utiliza o conceito de conjuntos de itens frequentes (\textit{frequent itemset}).

O FIHC está na classe dos algoritmos hierárquicos aglomerativos e baseia-se no seguinte critério de similaridade para a construção da árvore de grupos:

\begin{equation}
Sim(Ci \gets Cj) = \frac{Score(Ci \gets doc(Cj))}{N} + 1
\label{eqn:Sim}
\end{equation}

Onde \textit{Ci} e \textit{Cj} são os grupos usados na comparação de similaridade, \textit{Score} é uma medida de relevância de um documento em um grupo;

\begin{equation}
N = \sum_{x}tfidf(x, doc(Cj)) + \sum_{x'}tfidf(x', doc(Cj))
\label{eqn:Sim_N}
\end{equation}

e \textit{tfidf} é uma medida para avaliação da relevância de uma palavra em um documento.

\end{block}

%----------------------------------------------------------------------------------------

\end{column} % End of column 2.1

\begin{column}{\onecolwid}\vspace{-.6in} % The second column within column 2 (column 2.2)

%----------------------------------------------------------------------------------------
%	METHODS
%----------------------------------------------------------------------------------------

\begin{block}{Conjunto de itens frequentes}
Um conceito importante para o entendimento do FIHC é a noção de conjunto de itens frequentes de uma coleção, cuja definição é: um conjunto de palavras que ocorrem numa quantidade de documentos da coleção acima de um limiar de suporte definido pelo usuário.
\end{block}

\begin{block}{Arquitetura da biblioteca}

A biblioteca abrange todos os passos de uma solução para agrupamento, desde o pré-processamento dos documentos até os algoritmos de agrupamento propriamente ditos. Além disso, foi arquitetada para trabalhar com diversos idiomas. A sua arquitetura é mostrada na figura 2.

\begin{figure}
\includegraphics[width=0.95\linewidth]{arquitetura.png}
\caption{Arquitetura proposta para a biblioteca}
\end{figure}

O pré-processamento consiste nos seguintes passos:

\begin{itemize}
\item Detecção de idioma: O primeiro passo para realizar o agrupamento é identificar o idioma utilizado nos documentos, pois os algoritmos dos passos seguintes necessitam desse conhecimento. %Detecta o idioma mais provável da coleção de documentos atráves de método n-grama.
\item Tokenização: Segmenta os textos em palavras.
\item Limpeza: Remove as palavras que possuem pouca relevância no texto, como as preposições, os artigos e marcações gráficas.
\item \textit{Stemming}: Unifica formas variantes de palavras que possuem o mesmo significado, como as palavras ``economia'' e ``econômico''.
\end{itemize}

% Exemplo de negrito \textbf{Donec}

%\end{block}

%----------------------------------------------------------------------------------------
%	RESULTS
%----------------------------------------------------------------------------------------

%\begin{block}{O sistema hVINA}

Já o sistema \textit{hVINA} tem como objetivo criar uma interface amigável para que qualquer usuário possa utilizar a biblioteca, permitindo ao usuário informar uma coleção de artigos de seu interesse ou utilizar coleções pré-selecionadas pelo sistema.


\end{block}

%----------------------------------------------------------------------------------------

\end{column} % End of column 2.2

\end{columns} % End of the split of column 2 - any content after this will now take up 2 columns width

\end{column} % End of the second column

\begin{column}{\sepwid}\end{column} % Empty spacer column

\begin{column}{\onecolwid} % The third column

%----------------------------------------------------------------------------------------
%	CONCLUSION
%----------------------------------------------------------------------------------------

\begin{block}{Conclusão}

A biblioteca desenvolvida é modular, permitindo acrescentar o tratamento de outros idiomas ou implementar novos algoritmos de forma não intrusiva. Ademais, outros projetos além do \textit{hVINA} podem utilizá-la facilmente, visto que ela segue as especificações do distribuidor de pacotes do Ruby (RubyGems).

\begin{figure}
\includegraphics[width=1\linewidth]{hvina.png}
\caption{Tela principal do hVINA}
\end{figure}

O resultado obtido pela biblioteca em conjunto com o sistema \textit{hVINA} demostra a viabilidade de uma solução que tenha boa usabilidade e seja amigável a qualquer usuário, onde não haja a necessidade de treinamento do algoritmo e a configuração seja quase zero.

Porém, melhorias na escalabilidade da biblioteca devem ser feitas nos trabalhos futuros.

\end{block}

%----------------------------------------------------------------------------------------
%	REFERENCES
%----------------------------------------------------------------------------------------

\begin{block}{Referências}

\nocite{*} % Insert publications even if they are not cited in the poster
\small{\bibliographystyle{unsrt}
\bibliography{sample}}

\end{block}

%----------------------------------------------------------------------------------------
%	ACKNOWLEDGEMENTS
%----------------------------------------------------------------------------------------

\begin{center}
\begin{tabular}{ccc}
\includegraphics[width=0.12\linewidth]{ruby.png} & \includegraphics[width=0.12\linewidth]{IME.png} & \includegraphics[width=0.25\linewidth]{USP.jpg}
\end{tabular}
\end{center}

%----------------------------------------------------------------------------------------

\end{column} % End of the third column

\begin{column}{\sepwid}\end{column} % Empty spacer column

\end{columns} % End of all the columns in the poster

\end{frame} % End of the enclosing frame

\end{document}
